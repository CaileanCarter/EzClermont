\textit{Escherichia coli} is among the most widely studied organisms, and the species is very diverse\cite{Lukjancenko2010, Selander1987}.
%In addition to being a member of the human gut microbiome, \textit{E. coli} are found in soil, water, and livestock. The majority of lineages are harmless, but some can cause serious illness.  Thus, detecting and identifying the different lineages has remainedva valuable goal both for understanding the pathalogy as well as for detecting problematic strains.
Because of this diversity, many methods have been developed to differentiate the different  \textit{E. coli} lineages.  In 1987, Selandar  and colleagues used electrophoretic analysis of a 35 enzyme digest to classify the \textit{E. coli} Reference Collection (ECOR) in 6 phylogenetic groups (A-F)\cite{Selander1987}.  Clermont and colleagues published their triplex PCR method\cite{Clermont2000} of phylotyping, which proved to be an extremely valuable tool to differentiate groups A, B1, B2, and D, being cited over 625 times as of April 2018.  In 2013, Clermont and colleagues published an update to this work\cite{Clermont2013}, in which they showed that adding a 4th set of primers achieved higher resolution by expanded the method to detect groups E, F; additional primers were identified to differentiate the cryptic clades.  Most recently, the method was externed further to include a pair of primer to differentiate the new G phylogroup. This approach has been widely adopted as the method is reliable, easy to interpret, can correctly classify about 95\% of \textit{E. coli} strains, and can be performed rapidly.

Other typing schemes have been developed to classify \textit{E. coli} strains. These include the Achtman 7 gene Multi Locus Sequence Typing (MLST)\cite{Achtman2012, Alikhan2018}, Michigan EcMLST\cite{Qi2004}, whole-genome MLST (\url{http://www.applied-maths.com/applications/wgmlst}), core-genome MLST\cite{DeBeen2015}, two-locus MLST\cite{Weissman2012}, and ribosomal MLST\cite{Jolley2012}. All these sequencing-based methods classify \textit{E. coli} with greater accuracy and granularity than PCR-based phylotyping, but at the cost of simplicity: in addition to not requiring seqeuncing, it is easier to discuss a small set of phylotypes compared to the results of typing methods aimed at capturing greater genomic diversity. As a result, the Clermont 2013 phylotyping scheme remains a popular tool for \textit{E. coli} classification.

We developed EzClermont to provide a simple \textit{in silico} implementation of the Clermont phylotyping algorithm for genome assemblies. We have implemented the software as a web application and as a command-line tool for those needing to process large numbers of assemblies.
